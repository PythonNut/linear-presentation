\documentclass{fkpaper}

% Command for wrapping things with きっこ brackets (requires package
% kana.sty)
%
% Named ``np'' because I couldn't think of anything better than just
% reversing ``pn''
\usepackage{kana}
\newcommand{\np}[1]{\hspace{-.55em}〔#1〕\hspace{-.55em}}

\title{AN EFFICIENT ALGORITHM FOR CONSTRUCTING LINEAR KNOT DIAGRAMS}
\author{Jonathan Hayase and Forest Kobayashi}
\affiliation{Department of Mathematics, Harvey Mudd College,
  Claremont, CA, 91711}



\begin{document}

% ----------------------------- Title ----------------------------- %



\begin{abstract}
  \textbf{Abstract} Nullam eu ante vel est convallis dignissim. Fusce
  suscipit, wisi nec facilisis facilisis, est dui fermentum leo, quis
  tempor ligula erat quis odio. Nunc porta vulputate tellus. Nunc
  rutrum turpis sed pede. Sed bibendum. Aliquam posuere. Nunc aliquet,
  augue nec adipiscing interdum, lacus tellus malesuada massa, quis
  varius mi purus non odio. Pellentesque condimentum, magna ut
  suscipit hendrerit, ipsum augue ornare nulla, non luctus diam neque
  sit amet urna. Curabitur vulputate vestibulum lorem. Fusce sagittis,
  libero non molestie mollis, magna orci ultrices dolor, at vulputate
  neque nulla lacinia eros. Sed id ligula quis est convallis tempor.
  Curabitur lacinia pulvinar nibh. Nam a sapien.
\end{abstract}


\begin{multicols}{2}
  \section{Introduction}
  Lorem ipsum dolor sit amet, consectetur adipiscing elit, sed do
  eiusmod tempor incididunt ut labore et dolore magna aliqua. Ut enim
  ad minim veniam, quis nostrud exercitation ullamco laboris nisi ut
  aliquip ex ea commodo consequat. Duis aute irure dolor in
  reprehenderit in voluptate velit esse cillum dolore eu fugiat nulla
  pariatur. Excepteur sint occaecat cupidatat non proident, sunt in
  culpa qui officia deserunt mollit anim id est laborum.


  % We'll begin by describing some of common pieces of vocabulary used in
  % computability theory.


\end{multicols}


\bibliographystyle{abbrv}
\bibliography{bibbibbib}
\nocite{*}


\end{document}
